\documentclass[a4paper]{scrartcl}
% --- packages ---
\usepackage{amsmath}
\usepackage{amssymb}
\usepackage[ngerman]{babel}
\usepackage[backend=biber,style=ieee,sorting=none]{biblatex}
    \addbibresource{references.bib}
\usepackage{booktabs}
\usepackage{braket}
\usepackage[font=small,labelfont=bf]{caption}
\usepackage{csquotes}
\usepackage{float}
\usepackage{fontspec}
    \setmainfont[Ligatures=TeX]{Tex Gyre Pagella}
    \setmonofont[Ligatures=TeX]{Anonymous Pro}
\usepackage{graphicx}
\usepackage[italic]{hepnicenames}
\usepackage[pdfusetitle,unicode]{hyperref}
\usepackage{mathtools}
\usepackage{microtype}
\usepackage{minted}
\usepackage{subcaption}
\usepackage[math-style=ISO,bold-style=ISO]{unicode-math}
        \setmathfont{Tex Gyre Pagella Math}
\usepackage{xfrac}

% --- options ---
\setlength{\parindent}{0pt}  % no stupid indentation

% fix biblatex stuff (add collaboration, et al. etc)
\DeclareSourcemap{\maps[datatype=bibtex,overwrite=true]{\map{\step[fieldsource=Collaboration, final=true]\step[fieldset=usera, origfieldval, final=true]}}}
\renewbibmacro*{author}{\iffieldundef{usera}{\printnames{author}}{\printfield{usera}\addcomma\addspace\printnames{author}}}

% --- commands ---
\DeclarePairedDelimiter{\abs}{\lvert}{\rvert}
\DeclarePairedDelimiter{\mean}{\langle}{\rangle}
\renewcommand{\vec}[1]{\boldsymbol{#1}}
\renewcommand{\i}{\mathrm{i}}
\DeclareRobustCommand{\e}{\ensuremath{\mathrm{e}}}


\subject{Funktionale Programmierung WS 2013/14}
\title{Übungsblatt 5}
\author{Kevin Dungs\thanks{kevin.dungs@udo.edu}}

\begin{document}
\maketitle


\section*{Aufgabe 5.1}
\begin{minted}{haskell}
span (\y = y /= '5') :: [Char] -> ([Char], [Char])
head . map (\c -> c + 1) . takeWhile (\c -> c < 5) :: [Int] -> Int
map (map (\x -> 5 * x)) :: [[Int]] -> [[Int]]
\end{minted}


\section*{Aufgabe 5.2}
\begin{minted}{haskell}
    sumr [1, 2, 3, 4]
\end{minted}
wird folgendermaßen ausgeführt:
\begin{align*}
    4 &+ 0 & [1, 2, 3, 4] \\
    3 &+ 4 & [1, 2, 3] \\
    2 &+ 7 & [1, 2] \\
    1 &+ 9 & [1] \\
    10
\end{align*}

während
\begin{minted}{haskell}
    suml [1, 2, 3, 4]
\end{minted}
wie folgt ausgeführt wird:
\begin{align*}
    0 &+ 1 & [1, 2, 3, 4] \\
    1 &+ 2 & [2, 3, 4] \\
    3 &+ 3 & [3, 4] \\
    6 &+ 4 & [4] \\
    10
\end{align*}


\section*{Aufgabe 5.3}
\paragraph{1)}
\begin{minted}{haskell}
foldFor :: (Enum b) => a -> b -> b -> (a -> b -> a) -> a
foldFor x0 i0 i1 f = foldl f x0 [i0 .. i1]
\end{minted}

\paragraph{2)}
\begin{minted}{haskell}
foldWhile :: (Enum b) => (a -> b -> a) -> (b -> Bool) -> a -> b -> a
foldWhile f a x0 i0 = foldl f x0 $ takeWhile a [i0..]
\end{minted}


\section*{Aufgabe 5.4}
\inputminted{haskell}{binary.hs}


\section*{Aufgabe 5.5}
\inputminted{haskell}{zipper.hs}

\end{document}
