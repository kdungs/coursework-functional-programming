\documentclass[a4paper]{scrartcl}
% --- packages ---
\usepackage{amsmath}
\usepackage{amssymb}
\usepackage[ngerman]{babel}
\usepackage[backend=biber,style=ieee,sorting=none]{biblatex}
    \addbibresource{references.bib}
\usepackage{booktabs}
\usepackage{braket}
\usepackage[font=small,labelfont=bf]{caption}
\usepackage{csquotes}
\usepackage{float}
\usepackage{fontspec}
    \setmainfont[Ligatures=TeX]{Tex Gyre Pagella}
    \setmonofont[Ligatures=TeX]{Anonymous Pro}
\usepackage{graphicx}
\usepackage[italic]{hepnicenames}
\usepackage[pdfusetitle,unicode]{hyperref}
\usepackage{mathtools}
\usepackage{microtype}
\usepackage{minted}
\usepackage{subcaption}
\usepackage[math-style=ISO,bold-style=ISO]{unicode-math}
        \setmathfont{Tex Gyre Pagella Math}
\usepackage{xfrac}

% --- options ---
\setlength{\parindent}{0pt}  % no stupid indentation

% fix biblatex stuff (add collaboration, et al. etc)
\DeclareSourcemap{\maps[datatype=bibtex,overwrite=true]{\map{\step[fieldsource=Collaboration, final=true]\step[fieldset=usera, origfieldval, final=true]}}}
\renewbibmacro*{author}{\iffieldundef{usera}{\printnames{author}}{\printfield{usera}\addcomma\addspace\printnames{author}}}

% --- commands ---
\DeclarePairedDelimiter{\abs}{\lvert}{\rvert}
\DeclarePairedDelimiter{\mean}{\langle}{\rangle}
\renewcommand{\vec}[1]{\boldsymbol{#1}}
\renewcommand{\i}{\mathrm{i}}
\DeclareRobustCommand{\e}{\ensuremath{\mathrm{e}}}


\subject{Funktionale Programmierung WS 2013/14}
\title{Übungsblatt 3}
\author{Kevin Dungs\thanks{kevin.dungs@udo.edu}}

\begin{document}
\maketitle

\section*{Aufgabe 3.1}
\paragraph{1)}
\begin{minted}{haskell}
    Int -> (Bool, a) -> (a, Int)
  = (Int -> (Bool, a)) -> (a, Int)
\end{minted}

\paragraph{2)}
\begin{minted}{haskell}
    Int -> Bool -> (Int, Int) -> Int
  = ((Int -> Bool) -> (Int, Int)) -> Int
\end{minted}

\paragraph{3)}
\begin{minted}{haskell}
    a -> (a -> b -> c) -> b -> a
  = ((a -> ((a -> b) -> c)) -> b) -> a
\end{minted}


\section*{Aufgabe 3.2}
\paragraph{1)}
\begin{minted}{haskell}
    (\f -> f . f) $ \x -> (+3) x
  = \x -> ((+3) . (+3)) x
  = x + 6
\end{minted}

\paragraph{2)}
\begin{minted}{haskell}
    (\x -> 4 * x) . (\y -> 3 + y)
  = \x -> 4 * (3 + x)
\end{minted}


\section*{Aufgabe 3.3}
\paragraph{1)}
\begin{minted}{haskell}
    \(f, g) p x -> if p x then f x else g x
 :: ((a -> b), (a -> b)) -> (a -> Bool) -> a -> b
\end{minted}

\paragraph{2)}
\begin{minted}{haskell}
    (curry (\(x, y) -> if x < y then y else x)) 3
 :: Int -> Int
\end{minted}


\section*{Aufgabe 3.4}
\inputminted{haskell}{update.hs}

\end{document}
