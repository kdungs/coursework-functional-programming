\documentclass[a4paper]{scrartcl}
% --- packages ---
\usepackage{amsmath}
\usepackage{amssymb}
\usepackage[ngerman]{babel}
\usepackage[backend=biber,style=ieee,sorting=none]{biblatex}
    \addbibresource{references.bib}
\usepackage{booktabs}
\usepackage{braket}
\usepackage[font=small,labelfont=bf]{caption}
\usepackage{csquotes}
\usepackage{float}
\usepackage{fontspec}
    \setmainfont[Ligatures=TeX]{Tex Gyre Pagella}
    \setmonofont[Ligatures=TeX]{Anonymous Pro}
\usepackage{graphicx}
\usepackage[italic]{hepnicenames}
\usepackage[pdfusetitle,unicode]{hyperref}
\usepackage{mathtools}
\usepackage{microtype}
\usepackage{minted}
\usepackage{subcaption}
\usepackage[math-style=ISO,bold-style=ISO]{unicode-math}
        \setmathfont{Tex Gyre Pagella Math}
\usepackage{xfrac}

% --- options ---
\setlength{\parindent}{0pt}  % no stupid indentation

% fix biblatex stuff (add collaboration, et al. etc)
\DeclareSourcemap{\maps[datatype=bibtex,overwrite=true]{\map{\step[fieldsource=Collaboration, final=true]\step[fieldset=usera, origfieldval, final=true]}}}
\renewbibmacro*{author}{\iffieldundef{usera}{\printnames{author}}{\printfield{usera}\addcomma\addspace\printnames{author}}}

% --- commands ---
\DeclarePairedDelimiter{\abs}{\lvert}{\rvert}
\DeclarePairedDelimiter{\mean}{\langle}{\rangle}
\renewcommand{\vec}[1]{\boldsymbol{#1}}
\renewcommand{\i}{\mathrm{i}}
\DeclareRobustCommand{\e}{\ensuremath{\mathrm{e}}}


\subject{Funktionale Programmierung WS 2013/14}
\title{Übungsblatt 2}
\author{Kevin Dungs\thanks{kevin.dungs@udo.edu}}

\begin{document}
\maketitle

\section*{Aufgabe 2.1}
\paragraph{1)}
\begin{minted}{haskell}
    (\x -> 4 * x) $ (\y -> 3 + y) 1
  = (\x -> 4 * x) $ 4
  = 4 * 4 = 16
\end{minted}

\paragraph{2)}
\begin{minted}{haskell}
    (\x -> 4 + x + (\x -> 3 + x) 3) 1
  = 4 + 1 + (\x -> 3 + x) 3
  = 5 + (\x -> 3 + x)
  = 5 + 3 + 3 = 11
\end{minted}

\paragraph{3)}
\begin{minted}{haskell}
    (\f -> (f . f) 2) (*3)
  = ((*3) . (*3)) 2
  = 9 * 2 = 18
\end{minted}

\paragraph{4)}
\begin{minted}{haskell}
    (\x f -> x + f x) 5 (*3)
  = (\f -> 5 + f 5) (*3)
  = 5 + 3 * 5 = 20
\end{minted}


\section*{Aufgabe 2.2}
\paragraph{1}
\begin{minted}{haskell}
    \x y -> if x y then y else y + 2 :: (Int -> Bool) -> Int -> Int
\end{minted}

\paragraph{2)}
\begin{minted}{haskell}
    \f -> (f . f) 2 :: (Int -> Int) -> Int
\end{minted}

\paragraph{3)}
\begin{minted}{haskell}
    flip (.) :: (a -> b) -> (b -> c) -> a -> c
\end{minted}

\paragraph{4)}
\begin{minted}{haskell}
    uncurry (+) :: (Int, Int) -> Int
\end{minted}


\section*{Aufgabe 2.3}
\inputminted{haskell}{collatz.hs}

\end{document}
